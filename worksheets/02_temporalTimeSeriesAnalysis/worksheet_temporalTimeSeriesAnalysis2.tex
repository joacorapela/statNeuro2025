\documentclass[12pt]{article}

\usepackage{graphicx}
\usepackage[shortlabels]{enumitem}
\usepackage[colorlinks=]{hyperref}
\usepackage[margin=2cm]{geometry}
\usepackage{amsmath}
\usepackage{amssymb}

\title{Worksheet: temporal time series analysis (part II)}
\author{Joaquin Rapela}

\begin{document}

\maketitle

\begin{enumerate}

    \item The goal of this forecasting exercise is to produce a figure similar
        to that in slide \emph{Example (Forecasting with an AR(1) model)}, but
        for an AR(7) model.

        Simulate $N=10,000$ samples from an AR(7) model with parameters

        \begin{align*}
            \phi&=[5.0/6,-1.0/6,0.5/6,-0.25/6,0.5/6,-0.1/6,0.05/6]\\
            \sigma&=5.0
        \end{align*}


        \noindent Use the $m=500$ samples before the last sample to forecast $h=50$ samples into
        the future.

        You many want to use the script
        \href{https://github.com/joacorapela/statNeuro2025/blob/master/worksheets/02\_temporalTimeSeriesAnalysis/plot\_arForecastingYWCor.py}{plot\_arForecastingYWCor}
        and complete the code for the function
        \href{https://github.com/joacorapela/statNeuro2025/blob/6cfd79b15e698f4522eec97017f5471a4ff67633/worksheets/02\_temporalTimeSeriesAnalysis/tsAnalysisUtils.py\#L53}{forecast}
        in the module
        \href{https://github.com/joacorapela/statNeuro2025/blob/master/worksheets/02\_temporalTimeSeriesAnalysis/tsAnalysisUtils.py}{tsAnalysisUtils.py}.

        You should obtain a plot similar to that in
        Figure~\ref{fig:forecasting}.

        \begin{figure}
            \begin{center}
                \href{}{\includegraphics[width=4.5in]{../../../solutions/examples/sphinx-gallery/02_temporalTimeSeriesAnalysis/figures/forecastingAR7.png}}
            \end{center}
            \caption{Forecasting result for an AR(7) model}
            \label{fig:forecasting}
        \end{figure}

    \item Reproduce the figure in the slide \emph{Estimate coefficients of
        AR(3) model using the Yule-Walker estimators} in
        \href{https://github.com/joacorapela/statNeuro2025/blob/master/lectures/02_temporalTimeSeriesAnalysis/temporalTimeSeriesAnalysis.pdf}{lecture 2}. You may want to use the
        sample script available
        \href{https://github.com/joacorapela/statNeuro2025/blob/master/worksheets/02_temporalTimeSeriesAnalysis/plot_estimateCoefsAR3YW.py}{here}.
        To run this code, you will need to complete the function
        \href{https://github.com/joacorapela/statNeuro2025/blob/5c570ed3bbb4311d80c280fcb30c4a1ec26f5b53/worksheets/02_temporalTimeSeriesAnalysis/tsAnalysisUtils.py\#L28}{estimateCoefsAndNoisVarARpYW} in
        the module
        \href{https://github.com/joacorapela/statNeuro2025/blob/master/worksheets/02_temporalTimeSeriesAnalysis/tsAnalysisUtils.py\#L28}{tsAnalysisUtils.py},
        imported in the previous script.

        Hints:

        \begin{description}

            \item[solving the system of equations $A\mathbf{x}=\mathbf{b}$ in
                Numpy] to estimate the vector $\mathbf{x}$ that best
                approximates the previous equations in Numpy you can use
                \texttt{x = np.linealg.solve(A, \textbf{x})}.

            \item[computing inner products $\mathbf{x}^\intercal\mathbf{y}$ in
                Numpy] to calculate the previous inner product between vector
                $\mathbf{x}$ and $\mathbf{y}$ in Numpy you can use
                \texttt{np.inner(x, y)}.

            \item[computing the inverse $A^{-1}$ of matrix $A$ in Numpy] to
                calculate the previous inverse of matrix $A$ in Numpy you can
                use \texttt{np.linalg.inv(A)}.

        \end{description}

\end{enumerate}

\end{document}
