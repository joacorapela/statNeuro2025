\documentclass[12pt]{article}

\usepackage{graphicx}
\usepackage[shortlabels]{enumitem}
\usepackage[colorlinks=]{hyperref}
\usepackage[margin=2cm]{geometry}
\usepackage{amsmath}
\usepackage[dvipsnames]{xcolor}
\usepackage{listings}
\usepackage{xcolor}

\lstset{
    language=Python,
    basicstyle=\ttfamily\small,
    keywordstyle=\color{blue}\bfseries,
    stringstyle=\color{red},
    commentstyle=\color{green!60!black},
    numbers=left,
    numberstyle=\tiny\color{gray},
    stepnumber=1,
    numbersep=5pt,
    backgroundcolor=\color{white},
    showspaces=false,
    showstringspaces=false,
    showtabs=false,
    frame=single,
    tabsize=4,
    captionpos=b,
    breaklines=true,
    breakatwhitespace=false,
}

\title{Solution for the worksheet\\
\href{https://github.com/joacorapela/statNeuro2025/blob/master/worksheets/02_temporalTimeSeriesAnalysis/worksheet_temporalTimeSeriesAnalysis2.pdf}{temporal
time series analysis part II}}
\author{Joaquin Rapela}

\begin{document}

\maketitle

\begin{enumerate}

    \item Listing~\ref{lst:forecast} shows the completed function
        \emph{forecat} and Figure~\ref{fig:forecastAR7} shows the forecasting
        results for an AR(7) model.

        \begin{lstlisting}[language=Python,caption={completed \emph{forecast} function}\label{lst:forecast}]
def forecast(x, acov, mu, m, max_h):
    Gamma_m = buildGamma(acov=acov, m=m)
   	forecasts_means = np.empty(max_h, dtype=np.double)
    forecasts_vars = np.empty(max_h, dtype=np.double)
    xMinusMu_mR = (x-mu)[::-1][:m]
    for h in range(1, max_h+1):
       	gamma_mh = acov[h:(h+m)]
        a_m = np.linalg.solve(Gamma_m, gamma_mh)
        forecasts_means[h-1] = mu + np.inner(a_m, xMinusMu_mR)
        forecasts_vars[h-1] = acov[0] - np.inner(a_m, gamma_mh)
    return forecasts_means, forecasts_vars
        \end{lstlisting}

        \begin{figure}
            \begin{center}
                \href{}{\includegraphics[width=5in]{../../../../solutions/examples/sphinx-gallery/02_temporalTimeSeriesAnalysis/figures/forecastingAR7.png}}
            \end{center}
            \caption{Forecasting result for an AR(7) model}
            \label{fig:forecastAR7}
        \end{figure}

    \item Listing~\ref{lst:estimateCoefsAndNoisVarARpYW} shows the completed
        function \emph{estimateCoefsAndNoisVarARpYW} and
        Figure~\ref{fig:coefsAR3} shows the estimated coefficients of an AR(3)
        model.

        \begin{lstlisting}[language=Python,caption={completed \emph{estimateCoefsAndNoisVarARpYW} function}\label{lst:estimateCoefsAndNoisVarARpYW}]
def estimateCoefsAndNoiseVarARpYW(acov, p, N):
    Gammap = buildGamma(acov=acov, m=p)
    gammaph = acov[1:]
    phiHat = np.linalg.solve(Gammap, gammaph)
    sigma2Hat = acov[0] - np.inner(phiHat, gammaph)
    phiCovHat = sigma2Hat/N * np.linalg.inv(Gammap)
    return phiHat, phiCovHat, sigma2Hat 
        \end{lstlisting}

        \begin{figure}
            \begin{center}
                \includegraphics[width=5in]{../../../../solutions/examples/sphinx-gallery/02_temporalTimeSeriesAnalysis/figures/coefficientsAR3N10000.png}
            \end{center}
            \caption{True and estimated coefficients for an AR(3) model.}
            \label{fig:coefsAR3}
        \end{figure}

\end{enumerate}

\end{document}
