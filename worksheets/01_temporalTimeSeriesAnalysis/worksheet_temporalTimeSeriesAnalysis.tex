\documentclass[12pt]{article}

\usepackage{graphicx}
\usepackage[shortlabels]{enumitem}
\usepackage[colorlinks=]{hyperref}
\usepackage[margin=2cm]{geometry}
\usepackage{amsmath}

\title{Worksheet: temporal time series analysis}
\author{Joaquin Rapela}

\begin{document}

\maketitle

\begin{enumerate}

    \item Is the $X_t$ random process in Eq.~\ref{eq:randomProcess} wide-sense
        stationary (WSS)?

        \begin{align}
            X_t & = \phi X_{t-1} + W_t\label{eq:randomProcess}
        \end{align}

        where $\{W_t\}$ is a white noise random process with variance
        $\sigma^2_w$. For this random process to be WSS, can the parameter
        $\phi$ take any value? Why or why not?

    \item Write code to generate the figures in the
        \href{https://github.com/joacorapela/statNeuro2025/blob/master/lectures/01_temporalTimeSeriesAnalysis/temporalTimeSeriesAnalysis.pdf}{lecture}
        slide titled \emph{Analytical and estimated autocovariance function for
        AR(1)}. Provide the code and the generated figures.

        Hint: you may want to modify the code in the solution of the
        \href{https://github.com/joacorapela/statNeuro2025/blob/master/lectures/01_temporalTimeSeriesAnalysis/temporalTimeSeriesAnalysis.pdf}{lecture}
        slide titled \emph{Analytical and estimated autocovariance function for MA}.

    \item (optional) For the random walk with drift model:

        \begin{enumerate}[(a)]

            \item Calculate the covariance function $\gamma(s, t)$ and use it
                to derive the variance function $var(t)$.

            \item To check that your variance function is correct, complete and
                execute \href{}{this} python script. It plots 100 samples of
                the random walk with drift model (colored traces), with the
                mean (solid line) and 95\%
                confidence bands (dashed lines). See
                Figure~\ref{fig:sammplesRandomWalkWithDrift}. At any time point
                (abscissa) you should observe that 95\% of the traces (5
                traces) are above the upper line or below the lower line.

            \item Is the random walk with drift process wide-sense stationary?

        \end{enumerate}
    \item (optional) Forecasting an AR(5) random process.

        \begin{enumerate}[a]

            \item Simulate $n=500$ samples from an AR(5) random process. Set
                the coefficients, $\phi_i$, of the AR(5) model to values of
                your choice.

                Hint: use $|\phi_i|<1$ to ensure stationary.

            \item Forecast $X_{n+1},\ldots,X_{n+h}$, with $h=25$ using the last
                $m=20$ simulated values.

                Hint: you may want to complete the example code provided
                \href{}{here}.

        \end{enumerate}

        Provide your completed code and the generated figures.


\end{enumerate}

\end{document}
