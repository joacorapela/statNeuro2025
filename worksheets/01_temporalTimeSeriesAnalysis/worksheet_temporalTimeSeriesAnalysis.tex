\documentclass[12pt]{article}

\usepackage{graphicx}
\usepackage[shortlabels]{enumitem}
\usepackage[colorlinks=]{hyperref}
\usepackage[margin=2cm]{geometry}
\usepackage{amsmath}

\title{Worksheet: temporal time series analysis}
\author{Joaquin Rapela}

\begin{document}

\maketitle

\begin{enumerate}

    \item Is the $x_t$ random process in Eq.~\ref{eq:randomProcess} wide-sense
        stationary (WSS)?

        \begin{align}
            x_t & = \phi x_{t-1} + w_t\label{eq:randomProcess}
        \end{align}

        where $\{w_t\}$ is a white noise random process with variance
        $\sigma^2_w$. For this random process to be WSS, can the parameter
        $\phi$ take any value? Why or why not?

    \item Write code to generate the figures in the
        \href{https://github.com/joacorapela/statNeuro2025/blob/master/lectures/01_temporalTimeSeriesAnalysis/temporalTimeSeriesAnalysis.pdf}{lecture}
        slide titled \emph{Analytical and estimated autocovariance function for
        AR(1)}. Provide the code and the generated figures.

        Hint: you may want to modify the code in the solution of the
        \href{https://github.com/joacorapela/statNeuro2025/blob/master/lectures/01_temporalTimeSeriesAnalysis/temporalTimeSeriesAnalysis.pdf}{lecture}
        slide titled \emph{Analytical and estimated autocovariance function for MA}.

    \item (optional) For the random walk with drift model:

        \begin{enumerate}[(a)]

            \item Calculate the covariance function $\gamma(s, t)$ and use it
                to derive the variance function $var(t)$.

            \item To check that your variance function is correct, complete and
                execute
                \href{https://github.com/joacorapela/statNeuro2025/blob/master/worksheets/01_temporalTimeSeriesAnalysis/plot_samplesRandomWalkWithDrift.py}{this}
                python script. It plots 100 samples of the random walk with
                drift model (coloured traces), with the mean (solid line) and
                95\% confidence bands (dashed lines). See
                Figure~\ref{fig:randomWalkWithDriftSamples}. At any time point
                (abscissa) you should observe that 95\% of the traces (5
                traces) are above the upper line or below the lower line.


                \begin{figure}
                    \begin{center}
                        \href{https://www.gatsby.ucl.ac.uk/~rapela/statNeuro/2025/lectures/01_temporalTimeSeriesAnalysis/figures/randomNoiseWithDriftSamples.html}{\includegraphics[width=4.00in]{../../../mySolutions/01_temporalTimeSeriesAnalysis/figures/randomNoiseWithDriftSamples.png}}
                        \caption{One hundred samples of a random walk with
                        drift process (colour traces). The solid line is the
                        mean of the random process and the dotted lines mark
                        the 95\% confidence interval. At any time point 95\% of
                        the samples (i.e., 5 samples) should lie above or below
                        the dotted lines. Click on the figure to see its
                        interactive version.}
                        \label{fig:randomWalkWithDriftSamples}
                    \end{center}
                \end{figure}

            \item Is the random walk with drift process wide-sense stationary?
                Why or why not?

        \end{enumerate}

\end{enumerate}

\end{document}
