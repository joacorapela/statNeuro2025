\documentclass[12pt]{article}

\usepackage{graphicx}
\usepackage[shortlabels]{enumitem}
\usepackage[colorlinks=]{hyperref}
\usepackage[margin=2cm]{geometry}
\usepackage{amsmath}
\usepackage[dvipsnames]{xcolor}

\title{Solution for the worksheet
\href{https://github.com/joacorapela/statNeuro2025/blob/master/worksheets/01_temporalTimeSeriesAnalysis/worksheet_temporalTimeSeriesAnalysis.pdf}{temporal time series analysis}}
\author{Joaquin Rapela}

\begin{document}

\maketitle

\begin{enumerate}

    \item The random process in Eq.~1 of the
        \href{https://github.com/joacorapela/statNeuro2025/blob/master/worksheets/01_temporalTimeSeriesAnalysis/worksheet_temporalTimeSeriesAnalysis.pdf}
        {worksheet} is an autoregressive model of order one, AR(1).

        As show in the slide \textbf{Mean function} containing \textbf{Example
        (Mean function of the autoregressive model of order 1)}, the mean of this AR(1)
        model is zero.

        When $|\phi|<1$, as show in the slide \textbf{Autovariance function}
        containing \textbf{Example (Autocovariance function of AR(1))}, the
        covariance of an AR(1) random process is $\gamma(t+h,
        t)=\phi^h\sigma_w^2\frac{1}{1-\phi^2}$.
        When $|\phi|\ge 1$, the summation $\sum_{i=0}^\infty\phi^{2i}$
        does not converge, and the covariance $\gamma(t+h, t)$ is not well defined.
        Therefore, only when $|\phi|<1$,  the variance of this AR(1) random process is
        $\sigma^2=\sigma_w^2\frac{1}{1-\phi^2}$.

        When $|\phi|<1$, because (1) its mean is constant, (2) its variance is also constant and
        (3) its covariance does not depend on $t$, the AR(1) random process in
        wide-sense stationary.

    \item The code to generate the figures in the
        \href{https://github.com/joacorapela/statNeuro2025/blob/master/lectures/01_temporalTimeSeriesAnalysis/temporalTimeSeriesAnalysis.pdf}{lecture}
        slide titled \textbf{Analytical and estimated autocovariance function for AR(1)}, and the generated figures are available
        \href{https://joacorapela.github.io/statNeuro2025/auto_examples/01_temporalTimeSeriesAnalysis/plot_autoregressiveAutocovariance.html#sphx-glr-auto-examples-01-temporaltimeseriesanalysis-plot-autoregressiveautocovariance-py}{here}.

    \item For the random walk with drift model:

        \begin{enumerate}[(a)]

            \item The covariance function $\gamma(s, t)$ is
                \begin{align*}
                    \gamma(t,s)&=E\{(x_t-\mu_t)(x_s-\mu_s)\}=E\left\{
                        \left(\left(\delta t+\sum_{i=0}^tw_i\right)-\delta t\right)
                        \left(\left(\delta s+\sum_{j=0}^sw_j\right)-\delta s\right)
                    \right\}\\
                    &=E\left\{
                        \left(\sum_{i=0}^tw_i\right)
                        \left(\sum_{j=0}^sw_j\right)
                    \right\}
                    =E\left\{
                        \sum_{i=0}^t\sum_{j=0}^sw_iw_j
                    \right\}\\
                &=\sum_{i=0}^t\sum_{j=0}^sE\{w_iw_j\}=\sum_{k=0}^{\min(t,s)}E\{w_k^2\}=\sum_{k=0}^{\min(t,s)}\sigma_w^2=\min(t,s)\;\sigma_w^2
                \end{align*}

                Then the variance function $var(t)$ is 
                \begin{align*}
                    var(t)=\gamma(t,t)&=t\sigma_w^2
                \end{align*}

            \item Figure~\ref{fig:randomWalkWithDriftSamples} plots 100 samples
                of a random walk with drift model, its mean function and its
                95\% confidence bounds. That approximately 5 samples escape the
                confidence bounds at any time point suggest that the calculated
                variance function is correct.

                \begin{figure}
                    \begin{center}
                        \href{https://www.gatsby.ucl.ac.uk/~rapela/statNeuro/2025/lectures/01_temporalTimeSeriesAnalysis/figures/randomNoiseWithDriftSamples.html}{\includegraphics[width=4.00in]{../figures/randomNoiseWithDriftSamples.png}}
                        \caption{One hundred samples of a random walk with
                        drift process (colour traces). The solid line is the
                        mean of the random process and the dotted lines mark
                        the 95\% confidence interval. At any time point 95\% of
                        the samples (i.e., 5 samples) should lie above or below
                        the dotted lines. Click on the figure to see its
                        interactive version.}
                        \label{fig:randomWalkWithDriftSamples}
                    \end{center}
                \end{figure}

            \item The random walk with drift process not wide-sense stationary
                because (1) when $\delta\neq 0$ the mean is not constant, (2)
                the variance is neither constant, and (3) the autocovariance
                function does not depend only on the time separation of its
                arguments.

        \end{enumerate}

\end{enumerate}

\end{document}
