\documentclass[12pt]{article}

\usepackage{graphicx}
\usepackage{tikz}
\usepackage[shortlabels]{enumitem}
\usepackage[colorlinks=]{hyperref}
\usepackage[margin=2cm]{geometry}
\usepackage{amsmath}
\usepackage{amssymb}

\title{Worksheet: spectral time series analysis}
\author{Joaquin Rapela}

\begin{document}

\maketitle

The first five problems will examine oscillatory activity in local field
potentials recorded from the infralimbic cortex and from the basolateral amygdala
of rats. These recordings are available in this
\href{https://github.com/tne-lab/cl-example-data}{repository}, and two
publications related to this data are cited in the repository's
\href{https://github.com/tne-lab/cl-example-data/README.md}{README.md}.

In the following exercises you will use a 45-minutes local field potential
recording
saved at a temporal resolution of 30,000 samples per second, i.e., 30~kHz, in the file
\emph{08102017/time\_data\_pre\_45sec.mat}
of the previous repository.
To speed up its processing, I downsampled this file
to a resolution of 3,000 samples per second, i.e., 3~kHz.
You can download the downsampled file from
\href{https://www.gatsby.ucl.ac.uk/~rapela/statNeuro/2025/lectures/03_spectralTimeSeriesAnalysis/data/time_data_pre_45sec_ds10_v6.mat}{here}.

\begin{enumerate}

    \item Plot data from one channel in this recording. You can adapt the
        Python script provided
        \href{https://github.com/joacorapela/statNeuro2025/blob/master/worksheets/03_spectralTimeSeriesAnalysis/doPlotData.py}{here}.

    \item Estimate and plot the sample mean and covariance function of the
        data plotted above. Does the data appear to be wide-sense stationary?

        Hint: you could estimate the sample mean and covariance function from
        different section of the data. If these estimates change substantially
        among different segments, the data is probably not wide-sense
        stationary.

    \item Estimate the spectral density using the periodogram method, as
        indicated in the section
        \href{https://mark-kramer.github.io/Case-Studies-Python/03.html#step-4-power-spectral-density-or-spectrum}{Step
        4: Power spectral density, or spectrum} of the lecture \emph{The Power
        Spectrum (Part 1)}. You may want to complete the code provided
        \href{https://github.com/joacorapela/statNeuro2025/blob/master/worksheets/03_spectralTimeSeriesAnalysis/doPeriodogram.py}{here}.

    \item The periodogram is a noisy estimator of the spectral
        density. Use the function
        \href{https://docs.scipy.org/doc/scipy/reference/generated/scipy.signal.welch.html#scipy.signal.welch}{scipy.signal.welch} to re-estimate
        the spectral density function using the Welch method, and check if this estimate is better
        than the periodogram one.

    \item Both the periodogram and the Welch method assume that the time series
        is stationary. Use the spectrogram, as described in the section
        \href{https://hub.2i2c.mybinder.org/user/mark-kramer-case-studies-python-vd53gjx8/notebooks/03.ipynb#the-spectrogram}{Step
        6: The spectrogram} of the lecture \emph{The Power Spectrum (Part 1)},
        to bypass this assumption, and to check if the spectral density changes
        with time.

    \item (optional) We measure the LFP in human motor cortex with an Utah array. It is
        known that this LFP only has an oscillation at 11~Hz (i.e.,
        $LFP(t)=\cos(\omega_0\,t)$ with $\omega_0=2\pi\,f_0$~rad/sec,
        $f_0=11$~Hz). However, when we sample this LFP at a frequency of 10~Hz
        (i.e., $\omega_s=2\pi\,f_s$~rad/sec, $f_s=10$~Hz) we only observe an
        oscillation at 1~Hz (Figure~\ref{fig:undersampledLFP}).

        \begin{figure}
            \begin{center}
                \includegraphics[width=6in]{figures/undersampledLFP.png}

                \caption{An LFP oscillating at 11~Hz (i.e., $LFP(t)=\cos(\omega_0\,t)$ with $\omega_0=2\pi\,f_0$~rad/sec,
                $f_0=11$~Hz) when sampled at a frequency of 10~Hz (i.e.,
                $\omega_s=2\pi\,f_s$~rad/sec, $f_s=10$~Hz) only displays an
                oscillation at 1~Hz. Use the sampling theorem to explain this
                observation.  Code to generate this figure appears
                \href{https://github.com/joacorapela/neuroinformatics24/blob/master/worksheets/02_LFPs_spectralAnalysis/code/doExUndersampledLFP.py}{here}.}

            \end{center}
            \label{fig:undersampledLFP}
        \end{figure}

        \begin{enumerate}[(a)]

            \item explain the appearance of the 1~Hz oscillation using the
                sampling theorem.

                Hints:

                \begin{itemize}

                    \item the Fourier transform of a cosine is
                        $\mathcal{FT}\{\cos(\omega_0 t)\}=\frac{1}{2}[\delta(\omega-\omega_0)+\delta(\omega+\omega_0)]$
                        and has the spectrum in the figure below.
                        \input{figFTcos}

                    \item replicate the above spectrum, as indicated
                        by the sampling theorem, with replicas at multiples
                        of the sampling frequency $\omega_s=2\pi\,f_s$
                        ($f_s=\frac{1}{T_s}=10~Hz$).

                    \item check if any of the above replicates adds signal at
                        1~Hz (i.e., $\omega=2\pi\,f$~rad/sec, $f=1$~Hz).
                \end{itemize}

                Note: to avoid this type of problems where low-frequency
                oscillations appear due to frequencies in the signal above the
                Nyquist frequency (i.e., half of the sampling frequency),
                signals are low pass filtered with an analog filter at the
                Nyquist frequency before being sampled. This filter is called
                an \textbf{antialiasing filter}. Analog filters do not
                generate aliasing.

            \item use a sampling frequency above the Nyquist rate (i.e.,
                $f_s>2f_N$ where $f_N$ is the largest frequency in the signal)
                and check that oscillations at 11~Hz appear in the sampled
                signal. You may want to use
                \href{https://github.com/joacorapela/neuroinformatics24/blob/master/worksheets/02_LFPs_spectralAnalysis/code/doExUndersampledLFP.py}{this}
                code.

            \item build another example of an LFP having an oscillation at a
                high frequency that when sampled at a frequency below the
                Nyquist rate generates an oscillation at a lower frequency. You
                can use
                \href{https://github.com/joacorapela/neuroinformatics24/blob/master/worksheets/02_LFPs_spectralAnalysis/code/doExUndersampledLFP.py}{this} code to verify that with your values of
                the LFP frequency and the sampling frequency an oscillation
                at a low frequency emerges.

    \end{enumerate}
\end{enumerate}

\end{document}
