\documentclass[12pt]{article}

\usepackage{graphicx}
\usepackage{tikz}
\usepackage[shortlabels]{enumitem}
\usepackage[colorlinks=]{hyperref}
\usepackage[margin=2cm]{geometry}
\usepackage{amsmath}
\usepackage{amssymb}
\usepackage{amsthm}
\usepackage{listings}
\usepackage[title]{appendix}

\newtheorem{definition}{Definition}
\newtheorem{lemma}{Lemma}
\newtheorem{theorem}{Theorem}

\lstset{
    basicstyle=\fontsize{7}{9}\selectfont\ttfamily
}

\title{Worksheet: dimensionality reduction}
\author{Joaquin Rapela}

\begin{document}

\maketitle

This worksheet is a modification of a
\href{https://drive.google.com/file/d/1r90rlJpFKilQmNj1h27gmZdyi-RJeCq_/view}{worksheet}
from the 2023 edition of the
\href{https://www.ucl.ac.uk/cortexlab/neuroinformatics-class-page}{NEUO019
Foundations of Neuroinformatics} course, taught by
\href{https://profiles.ucl.ac.uk/31489}{Prof.~Kenneth Harris} at UCL.

We are going to use the singular value decomposition (SVD) to uncover hidden
structure in the firing rate of a population of neurons in a mouse motor cortex
and striatum.
%
We will use data from the
\href{https://www.internationalbrainlab.com/}{International Brain Laboratory}
(IBL). To learn more about the IBL please go to \url{https://viz.internationalbrainlab.org}.

\begin{enumerate}

    \item First, install package needed for this worksheet by typing the
        following in the command line. I recommend using conda environments. If
        you are doing so, before running the next command, activate your conda
        environment.

        \begin{lstlisting}[backgroundcolor=\color{lightgray}]
        cd <repository_directory>/worksheets/04_dimensionalityReduction
        git pull
        pip install -r requirements.txt
        \end{lstlisting}

        Next use the script
        \href{https://github.com/joacorapela/statNeuro2025/blob/master/worksheets/04_dimensionalityReduction/doEx1.py}{doEx1.py}
        to plot the firing rates of a population of neurons, as in
        Fig.~\ref{fig:ex1}.

        \begin{figure}
            \begin{center}
                \href{https://www.gatsby.ucl.ac.uk/~rapela/statNeuro/2025/worksheets/04_dimensionalityReduction/figures/binned_spikes_binSize_1.00_original.html}{\includegraphics[width=6in]{../../figures/binned_spikes_binSize_1.00_original.png}}
                \caption{Z-scores of binned spikes times of all unsorted
                neurons. Click on the image to see its interactive version.}
            \end{center}
            \label{fig:ex1}
        \end{figure}

        In this script we first import libraries and open a connection IBL public data
        server:

        \begin{lstlisting}[backgroundcolor=\color{lightgray},language=Python]
        import numpy as np
        import matplotlib.pyplot as plt
        import one.api
        import scipy
        \end{lstlisting}

        Next, we load information about the spikes and trials for one
        experiment. Each experiment has a unique experiment ID, which is a long
        string identifying the experiment. This particular one is an experiment
        made at New York University, where they recorded from the motor cortex
        and striatum.  We load two data objects: dictionaries containing
        information about the spikes, and about the trials.

        \begin{lstlisting}[backgroundcolor=\color{lightgray},language=Python]
        eID = 'ebe2efe3-e8a1-451a-8947-76ef42427cc9'
        spikes = one.load_object(eID, 'spikes', 'alf/probe00/pykilosort')
        trials = one.load_object(eID, 'trials')
        \end{lstlisting}

        We then group the spikes together into an array, using the function
        \texttt{np.histogramdd}. Also we Z-score the activity of each cell (set
        it to have mean 0 and std 1)

        \begin{lstlisting}[backgroundcolor=\color{lightgray},language=Python]
        eID = 'ebe2efe3-e8a1-451a-8947-76ef42427cc9'
        bin_size = 1
        spike_time_bin = np.floor(spikes.times/bin_size).astype(int)
        activity_array, _ = np.histogramdd((spike_time_bin,spikes.clusters), bins=( spike_time_bin.max(), spikes.clusters.max()))
        activity_array = activity_array.T % after transposing the shape of activity_array is n_neurons x n_times
        activity_arrayZ = stats.zscore(activity_array)
        \end{lstlisting}

        Finally we plot a pseudocolor image of the array, but just sorted in
        whatever order it came. The functions \texttt{getHeatmap} and
        \texttt{getHovertext} are provided in module
        \href{https://github.com/joacorapela/statNeuro2025/blob/master/worksheets/04_dimensionalityReduction/utils.py}{utils.py}.

        \begin{lstlisting}[backgroundcolor=\color{lightgray},language=Python]
        eID = 'ebe2efe3-e8a1-451a-8947-76ef42427cc9'
        zmin, zmax = np.percentile(activity_arrayZ, q=(1.0, 99.0))
        hovertext = utils.getHovertext(
            times=times, clusters_ids=clusters_ids, z=activity_arrayZ.T,
            channels_for_clusters=clusters.channels,
            regions_for_channels=els[probe_id]["acronym"])
        fig = utils.getHeatmap(xs=times, ys=clusters_ids, zs=activity_arrayZ.T,
                            hovertext=hovertext, zmin=zmin, zmax=zmax,
                            x_label=x_label, y_label=y_label,
                            colorbar_title=colorbar_title)
        fig.write_image(fig_filename_pattern.format(bin_size, "original",  "png"))
        fig.write_html(fig_filename_pattern.format(bin_size, "original",  "html"))
        fig.show()
        \end{lstlisting}

        \item We want to use the SVD to find the temporal activity pattern that
            captures most of the variance in the population (i.e., the
            population firing pattern), and sort the neurons in
            Fig.~\ref{fig:ex1} according to the similarity of their firing
            rates to the population firing pattern.

            Take a matrix $M\in\mathbb{R}^{m\times n}$, we can use
            the SVD to decompose $M$ into as a summation of $r=\min(m, n)$
            rank one matrices (a rank one matrix $M$ is one that is the
            product of a column vector $\mathbf{u}$ times a row vector
            $\mathbf{v}^\intercal$; i.e.,
            $M=\mathbf{u}\mathbf{v}^\intercal$). Eq.~\ref{eq:rank1Decomp}
            gives the decomposition of $M$ into a sum of $r$ rank one
            matrices, and Eq.~\ref{eq:rank1Approx} gives the best rank one
            approximation to $M$.

            \begin{align}
                M&=USV=\sum_{i=0}^{r-1}s_iU[:,i]V[:,i]^\intercal\label{eq:rank1Decomp}\\
                &\simeq s_0\,U[:,0]\,V[:,0]^\intercal\label{eq:rank1Approx}
            \end{align}

            For the z-scored firing rates we are analysing in this worksheet,
            the vector $V[:,0]$ gives the population firing pattern. When
            $U[0,i]$ is positive (negative) and has a large absolute value, the
            z-scored firing rate of the ith neuron will be highly correlated
            (anticorrelated) with the population firing pattern.

            Plot the z-scored firing rates, as in Fig.~\ref{fig:ex1}, but
            sorting the neurons according to the correlation of their firing
            pattern with population firing pattern.
            Add to this plot a vertical line at the time of the last response
            of the subject.
            You may want to complete
            the script
            \href{https://github.com/joacorapela/statNeuro2025/blob/master/worksheets/04_dimensionalityReduction/doEx2.py}{doEx2.py}.
            Fig.~\ref{fig:ex2} plots the sorted spikes rates
            that I obtained.

            \begin{figure}
                \begin{center}
                    \href{https://www.gatsby.ucl.ac.uk/~rapela/statNeuro/2025/worksheets/04_dimensionalityReduction/figures/binned_spikes_svd_binSize_1.00_u0Sorted.html}{\includegraphics[width=6in]{../../figures/binned_spikes_svd_binSize_1.00_u0Sorted.png}}
                    \caption{Z-scores of binned spikes times of all neurons.
                    Neurons have been sorted according to the similarity of
                    their firing pattern to the population firing pattern in
                    Fig.~\ref{fig:populationFiringPattern}. Click on the image to
                    see its interactive version.}
                \end{center}
                \label{fig:ex2}
            \end{figure}

        \item Fig.~\ref{fig:ex2} suggest that the response of the neurons in
            the population is related to the subjects response. To verify this,
            plot the population firing pattern and add to this plot vertical lines
            indicating subject response times. You may want to complete the
            script
            \href{https://github.com/joacorapela/statNeuro2025/blob/master/worksheets/04_dimensionalityReduction/doEx3.py}{doEx3.py}.
            Figure~\ref{fig:populationFiringPattern} shows the population firing
            pattern that I obtained. Note that the population firing rate
            pattern, in most cases, peaks shortly after subject response times.

            \begin{figure}
                \begin{center}
                    \href{https://www.gatsby.ucl.ac.uk/~rapela/statNeuro/2025/worksheets/04_dimensionalityReduction/figures/binned_spikes_svd_binSize_1.00_msTemporalPattern.html}{\includegraphics[width=6in]{../../figures/binned_spikes_svd_binSize_1.00_msTemporalPattern_segment.png}}
                    \caption{Segment of the population firing pattern (blue
                    trace) and response times of the mice (vertical
                    lines). Click on the image to see its interactive version.}
                \end{center}
                \label{fig:populationFiringPattern}
            \end{figure}

        \item (optional) As proved in the Eckart-Young-Mirsky theorem
            (Theorem~\ref{thm:eckart-young-mirsky}), the truncated rank $\nu$
            singular value decomposition of matrix $M$
            (Eq.~\ref{eq:truncatedSVD}), is the best rank $\nu$ approximation
            to $M$, among all possible rank $\nu$ matrices. In addition, the
            Frobenius error of this best approximation can be computed from the
            singular values of the SVD decomposition of matrix $M$, without
            having to assemble the best approximation.

            For $\nu=1,\ldots,5$, verify the above claim by assembling the
            truncated SVD, $M_\nu$, of the matrix, $M$, in Fig.\ref{fig:ex2}, computing the
            Frobenius norm of the difference between $M$ and $M_\nu$, and
            comparing this norm with the the sum of the first $\nu$ singular
            values of matrix $M$. Plot $M_\nu$ and write in the title the
            Frobenius error (i.e., empirical error) and sum of the squared
            first $\nu$
            singular values (i.e., the analytical error), as shown in
            Fig.~\ref{fig:truncatedSVD-rank1} for $\nu=1$.
            You may want to complete the
            script
            \href{https://github.com/joacorapela/statNeuro2025/blob/master/worksheets/04_dimensionalityReduction/doEx4.py}{doEx4.py}.
            The truncated SVDs for ranks 1, 2 and 5, with their empirical and
            analytical approximation errors are shown inf
            Figs.~\ref{fig:truncatedSVD-rank1}, \ref{fig:truncatedSVD-rank2},
            and~\ref{fig:truncatedSVD-rank5}, respectively.

            \begin{figure}
                \begin{center}
                    \href{https://www.gatsby.ucl.ac.uk/~rapela/statNeuro/2025/worksheets/04_dimensionalityReduction/figures/binned_spikes_truncatedSVD_binSize_1.00_nComponents_1.html}{\includegraphics[width=5in]{../../figures/binned_spikes_truncatedSVD_binSize_1.00_nComponents_1.png}}

                    \caption{Low-rank approximation of the image in
                    Figure~\ref{fig:ex2} using a truncated SVD or rank \i. The
                    title reports the empirical and analytical errors of the
                    reconstruction. The empirical error is the Frobenius norm
                    of the difference between the low-rank approximation and
                    the image in Figure~\ref{fig:ex2}. The analytical error is
                    computed from the singular values of the image in
                    Figure~\ref{fig:ex2} using Eq.~\ref{eq:errorFNorm}. Click
                    on the image to see its interactive version.}
                    \label{fig:truncatedSVD-rank1}
                \end{center}
            \end{figure}
            \foreach \i in {2,5}{
                \begin{figure}
                    \begin{center}

                        \href{http://www.gatsby.ucl.ac.uk/~rapela/neuroinformatics/2023/ws5/figures/binned_spikes_truncatedSVD_binSize_1.00_nComponents_\i.html}{\includegraphics[width=5in]{../../figures/binned_spikes_truncatedSVD_binSize_1.00_nComponents_\i.png}}

                        \caption{Low-rank approximation of the image in
                        Figure~\ref{fig:ex2} using a truncated SVD of rank
                        \i. Same format as that in Figure~\ref{fig:truncatedSVD-rank1}.
                        }

                        \label{fig:truncatedSVD-rank\i}

                    \end{center}
                \end{figure}
            }


\end{enumerate}

\pagebreak
\begin{appendices}

\section{Notes on the SVD}
\label{sec:notesOnTheSVD}

    \begin{definition}[The SVD]
        Given $M\in\mathbb{C}^{m\times n}$, a singular value decomposition (SVD)
        of $M$ is a factorisation:

        \begin{align*}
            M = USV^*
        \end{align*}

        where

        \begin{align*}
            U &\in \mathbb{C}^{m\times m}\quad\text{is unitary,}\\
            V &\in \mathbb{C}^{n\times n}\quad\text{is unitary,}\\
            S &\in \mathbb{C}^{m\times n}\quad\text{is diagonal.}
        \end{align*}

        In addition, it is assumed that the diagonal entries $s_k$ of $S$ are
        nonnegative and in nonincreasing order; that is, $s_1\ge
        s_2\ge\ldots\ge s_p\ge 0$, where $p=\min(m, n)$.
    \end{definition}

    \begin{definition}[Rank of a matrix]

        The column rank of a matrix is the dimension of the space spanned by
        its columns. Similarly, the row rank of a matrix is the dimension of
        the space spanned by its rows. The column rank of a matrix is always
        equal to its row rank. This is a corollary of the SVD. So we refer to
        this number simply as the rank of a matrix.

        \label{def:rank}
    \end{definition}

    The rank of a matrix can be interpreted as a measure of the complexity of
    the matrix. Matrices with lower rank are simpler than those with larger
    rank.

    The SVD decomposes a matrix as a sum of rank-one (i.e., very simple)
    matrices. 

    \begin{align*}
        M = \sum_{k=1}^rs_k\mathbf{u}_k\mathbf{v}_k^*
    \end{align*}

    There are multiple other decompositions as sums of rank-one matrices. If
    $M\in\mathbb{C}^{m\times n}$, then it can be decomposed as a sum of $m$
    rank-one matrices given by its rows (i.e.,
    $M=\sum_{i=1}^m\mathbf{e}_i\mathbf{m}_{i,\cdot}^*$, where $\mathbf{e}_i$ is
    the m-dimensional canonical unit vector, and $\mathbf{m}_{i,\cdot}$ is the ith row
    of $M$), or as a sum of $n$ rank-one matrices given by its columns (i.e.,
    $M=\sum_{j=1}^n\mathbf{m}_{\cdot,j}\mathbf{e}_j^*$, where $\mathbf{e}_j$ is
    the n-dimensional canonical unit vector, and $\mathbf{m}_{\cdot,j}$ is the jth
    column of $M$), or a sum of $mn$ rank-one matrices each containing only one
    non-zero element (i.e., $M=\sum_{i=1}^m\sum_{j=1}^nm_{ij}E_{ij}$, where
    $E_{ij}$ is the matrix with all entries equal to zero, except the $ij$
    entry that is one, and $m_{ij}$ is the entry of $M$ at position ij).

    A unique characteristic of the SVD compared to these other decompositions
    is that, if the rank of a matrix is $r$, then its SVD yields optimal
    approximations of lower rank $\nu$, for $\nu=1,\ldots,r$, as shown by
    Theorem~\ref{thm:eckart-young-mirsky}.

    \begin{definition}[Frobenius norm]
        The Frobenius norm of matrix $M\in\mathbb{C}^{m\times n}$ is

        \begin{align*}
            \|M\|_F=\left(\sum_{i=1}^m\sum_{j=1}^nm_{ij}^2\right)^{1/2}
        \end{align*}

    \end{definition}

    Note that

    \begin{align}
        \|M\|_F=\sqrt{tr(M^*M)}=\sqrt{tr(MM^*)}
        \label{eq:frobeniusAsTrace}
    \end{align}

    \begin{lemma}[Orthogonal matrices preserve the Frobenius norm]
        Let $M\in\mathbb{C}^{m\times n}$ and let $P\in\mathbb{C}^{m\times m}$
        and $Q\in\mathbb{C}^{n\times n}$ be orthogonal matrices. Then

        \begin{align*}
            \|PMQ\|_F=\|M\|_F
        \end{align*}

        \label{lemma:orthogonalPreserveF}
    \end{lemma}

    \begin{proof}
        \begin{align}
            \|PMQ\|_F&=\sqrt{tr((PMQ)(PMQ)^*)}=\sqrt{tr(PMQQ^*M^*P^*)}=\sqrt{tr(PMM^*P^*)\label{eq:frobInvLine1}}\\
                     &=\sqrt{tr(P^*PMM^*)}=\sqrt{tr(MM^*)}=\|M\|_F\label{eq:frobInvLine2}
        \end{align}
        Notes:
        \begin{enumerate}
            \item The first equality in Eq.~\ref{eq:frobInvLine1} follows
                Eq.~\ref{eq:frobeniusAsTrace}.
            \item The second equality in Eq.~\ref{eq:frobInvLine1} uses the fact
                that $(AB)^*=B^*A^*$.
            \item The third equality in Eq.~\ref{eq:frobInvLine1} holds because
                $Q$ is orthogonal (i.e., $QQ^*=I$).
            \item The first equality in Eq.~\ref{eq:frobInvLine2} uses the
                cyclic property of the trace (i.e., tr(ABC)=tr(CAB)).
            \item The first equality in Eq.~\ref{eq:frobInvLine2} holds by the
                orthogonality of $P$.
            \item The last equality in Eq.~\ref{eq:frobInvLine2} again applies
                Eq.~\ref{eq:frobeniusAsTrace}.
        \end{enumerate}
    \end{proof}

    A direct consequence of Lemma~\ref{lemma:orthogonalPreserveF} is that the
    Frobenius norm of any matrix $M=USV^*$ is

    \begin{align*}
        \|M\|_F=\|USV^*\|_F=\|S\|_F=\sqrt{\sum_{k=1}^rs_k^2}
    \end{align*}

    Another consequence of Lemma~\ref{lemma:orthogonalPreserveF} is 
    the error in approximating a matrix $M$ of rank $r$ with its truncated
    SVD of rank $\nu$ (i.e., $M_\nu=\sum_{k=1}^\nu s_k\mathbf{u}_k\mathbf{v}_k^*$) is

    \begin{align}
        \|M-M_\nu\|_F=\|\sum_{k=1}^rs_k\mathbf{u}_k\mathbf{v}_k^*-\sum_{k=1}^\nu
        s_k\mathbf{u}_k\mathbf{v}_k^*\|_F=\|\sum_{k={\nu+1}}^rs_k\mathbf{u}_k\mathbf{v}_k^*\|_F=\sqrt{\sum_{k=\nu+1}^rs_k^2}\label{eq:truncSVDerror}
    \end{align}

    \begin{theorem}[Eckart-Young-Mirsky]
        Let $M\in\mathbb{C}^{m\times n}$ be of rank r with singular value
        decomposition $M=\sum_{k=1}^rs_k\mathbf{u}_k\mathbf{v}_k^*$. For
        any $\nu$ with $0\leq\nu\leq r$, define

        \begin{align}
            M_\nu=\sum_{k=1}^\nu s_k\mathbf{u}_k\mathbf{v}_k^*
            \label{eq:truncatedSVD}
        \end{align}

        Then

        \begin{align}
            \|M-M_\nu\|_F=\inf_{\substack{\tilde{M}\in\mathbb{C}^{m\times n}\\\text{rank}(\tilde{M})\leq\nu}}\|M-\tilde{M}\|_F=\sqrt{\sum_{k=\nu+1}^rs_k^2}\label{eq:errorFNorm}
        \end{align}

        \label{thm:eckart-young-mirsky}
    \end{theorem}

    \begin{proof}
        We use the Weyl's inequality that relates the singular values of a sum
        of two matrices to the singular values of each of these matrices.
        Precisely, if $X,Y\in\mathbb{C}^{m\times n}$ and $s_i(X)$ is the ith
        singular value of $X$, then

        \begin{align}
            s_{i+j-1}(X+Y)\leq s_i(X)+s_j(Y)
            \label{eq:weylsInequality}
        \end{align}

        Let $\tilde{M}$ be a matrix of rank at most $\nu$. Applying
        Eq.~\ref{eq:weylsInequality} to $X=M-\tilde{M}$, $Y=\tilde{M}$ and
        $j-1=\nu$ we obtain

        \begin{align}
            s_{i+\nu}(M)\leq s_i(M-\tilde{M})+s_{\nu+1}(\tilde{M})=s_i(M-\tilde{M})\label{eq:svMandMerror}
        \end{align}

        The last equality in Eq.~\ref{eq:svMandMerror} holds because $\tilde{M}$
        has rank less or equal to $\nu$, and therefore its $\nu+1$ singular value is zero.

        \begin{align}
            \|M-M_\nu\|_F^2&=\sum_{j=\nu+1}^rs_j^2(M)=\sum_{i=1}^{r-\nu}s_{i+\nu}^2(M)\leq\sum_{i=1}^{r-\nu}s_i^2(M-\tilde{M})\leq\sum_{i=1}^{\min(m,n)}s_i^2(M-\tilde{M})\label{eq:final1}\\
                           &=\|M-\tilde{M}\|_F^2\label{eq:final2}
        \end{align}

        Notes:
        \begin{enumerate}
            \item The first equality in Eq.~\ref{eq:final1} holds by
                Eq.~\ref{eq:truncSVDerror}.
            \item The second equality in Eq.~\ref{eq:final1} used the change of
                variables $i=j-\nu$.
            \item The first inequality in Eq.~\ref{eq:final1} used
                Eq.~\ref{eq:svMandMerror}
            \item The last inequality in Eq.~\ref{eq:final1} is true because
                $r-\nu\leq\min(m,n)$ and adding squared singular values to the sum in
                the left hand side only increases this sum.
            \item The equality in Eq.~\ref{eq:final2} again holds by
                Eq.~\ref{eq:errorFNorm} and by the fact that singular values of
                index larger than the rank of a matrix are zero.
        \end{enumerate}

        The last equality in Eq.~\ref{eq:errorFNorm} follows from
        Eq.~\ref{eq:truncSVDerror}.

    \end{proof}

\end{appendices}

\end{document}
