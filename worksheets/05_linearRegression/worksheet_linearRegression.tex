\documentclass[12pt]{article}

\usepackage{graphicx}
\usepackage{natbib}
\usepackage{apalike}
\usepackage{listings}
\usepackage{amsmath}
\usepackage{amssymb}
\usepackage{amsthm}
\usepackage{xcolor}
\usepackage{subcaption}
\usepackage{float}
\usepackage{bm}

\usepackage[shortlabels]{enumitem}
\usepackage[colorlinks]{hyperref}
\usepackage[margin=2cm]{geometry}

\newtheorem{definition}{Definition}
\newtheorem{lemma}{Lemma}
\newtheorem{theorem}{Theorem}

\definecolor{codegreen}{rgb}{0,0.6,0}
\definecolor{codegray}{rgb}{0.5,0.5,0.5}
\definecolor{codepurple}{rgb}{0.58,0,0.82}
\definecolor{backcolour}{rgb}{0.95,0.95,0.92}

\lstdefinestyle{mystyle}{
    backgroundcolor=\color{backcolour},
    commentstyle=\color{codegreen},
    keywordstyle=\color{magenta},
    numberstyle=\tiny\color{codegray},
    stringstyle=\color{codepurple},
    basicstyle=\ttfamily\footnotesize,
    breakatwhitespace=false,
    breaklines=true,
    captionpos=b,
    keepspaces=true,
    numbers=left,
    numbersep=5pt,
    showspaces=false,
    showstringspaces=false,
    showtabs=false,
    tabsize=2
}

\lstset{style=mystyle}

\title{Worksheet: Linear regression for the estimation of nonlinear receptive
field models}
\author{Joaquin Rapela}

\begin{document}

\maketitle

Following~\citet{rapelaEtAl06}, to model responses of visual cell in animals to
stimulation with natural images, we
project these images into a small number of basis
functions (learned using the projection pursuit regression
algorithm~\citep{friedmanAndStuetzle81}), $\bm{\alpha}_1,\ldots,\bm{\alpha}_L$. Below we call \emph{relevant
dimensions} to these basis functions. We denote the projections of the nth
image, $\mathbf{v}_n$, into the lth relevant dimension, $\bm{\alpha}_l$, by
$x_{nl}=\bm{\alpha}_l^\intercal\mathbf{v}_n$. Then we model the number of spikes produced by the cell
in response to the nth image, $y_n$, with a multivariate polynomial of degree $D$.
Equation~\ref{eq:2RDsOrder3} shows a model of the response of
a cell to the nth image with
two basis functions, $L=2$, and a
multivariate polynomial of degree $D=3$.

\begin{align}
    y_n&=\beta_0+\beta_1x_{n1}+\beta_2x_{n2}+\nonumber\\
       &\quad\beta_{11}x_{n1}^2+2\beta_{12}x_{n1}x_{n2}+\beta_{22}x_{n2}^2+\nonumber\\
       &\quad\beta_{111}x_{n1}^3+3\beta_{112}x_{n1}^2x_{n2}+3\beta_{122}x_{n1}x_{n2}^2+\beta_{222}x_{n2}^3\nonumber\\
       &=[1,x_{n1},x_{n2},x_{n1}^2,2x_{n1}x_{n2},x_{n2}^2,x_{n1}^3,3x_{n1}^2x_{n2},3x_{n1}x_{n2}^2,x_{n2}^3]
         \raisebox{-5.6em}{$\left[\begin{array}{c}
             \beta_0\\\beta_1\\\beta_2\\\beta_{11}\\\beta_{12}\\\beta_{22}\\\beta_{111}\\\beta_{112}\\\beta_{122}\\\beta_{222}
                                \end{array}\right]$}
    \label{eq:2RDsOrder3}
\end{align}

Thus the above model is nonlinear on the input images, but linear on the
model parameters. To estimate these parameters we use the method of least squares.

In this worksheet you will

\begin{enumerate}

    \item do cross-validation to learn the optimal number of relevant dimensions,
        $L$, and the optimal degree of the multivariate polynomial, $D$.

    \item plot residuals to check the goodness of fit of the model.

    \item computer bootstrap confidence intervals to test the significance of
        model coefficients.

\end{enumerate}

It is a good practice to first test your methods with simulated data, for which
you know the ground truth. Thus, we will first perform this exercise on data from a
simulated cell and then on data recorded from a complex cell in primary visual
cortex of an anaesthetised cat~\citep{felsenEtAl05}.

\section{Simulated complex cell}

We simulated responses of a complex cell using the energy
model~\citep{adelsonAndBergen85} in Eq.~\ref{eq:energyModel}. This model
uses two identical Gabor relevant dimensions, $\bm{\alpha}_1$ and
$\bm{\alpha}_2$, that are $180^\circ$ out of phase. The relevant dimensions used for this simulation appear in Figure~\ref{fig:rdsSim}.

\begin{align}
    y_n=(\bm{\alpha}_1^\intercal\mathbf{v}_n)^2+(\bm{\alpha}_2^\intercal\mathbf{v}_n)^2
    \label{eq:energyModel}
\end{align}

\begin{figure}[H]
    \begin{center}
        \href{https://www.gatsby.ucl.ac.uk/~rapela/neuroinformatics/2023/ws6/figures/simCC.html}{\includegraphics[width=5.5in]{figures/simCC_rds.png}}

        \caption{Relevant dimensions used for the complex cell simulation.}

        \label{fig:rdsSim}
    \end{center}
\end{figure}

All the code for this worksheet appears in the subdirectory
\href{https://github.com/joacorapela/neuroinformatics24/blob/master/worksheets/06_linearRegression/code/scripts}{worksheets/06\_linearRegression/code/scripts} of the class
\href{https://github.com/joacorapela/neuroinformatics24}{repo}. In particular,
the code to simulate a complex cells is in script
\href{https://github.com/joacorapela/neuroinformatics24/blob/master/worksheets/06_linearRegression/code/scripts/doSimulateComplexCell.py}{doSimulateComplexCell.py}.

Prior to running this code install the required dependencies by running, from the subdirectory
\href{https://github.com/joacorapela/neuroinformatics24/blob/master/worksheets/06_linearRegression/code/scripts}{worksheets/06\_linearRegression/code/scripts},
the following commands:

\noindent\texttt{pip install -r requirements.txt}\linebreak
\noindent\texttt{pip install -r requirements2.txt}

The script
\href{https://github.com/joacorapela/neuroinformatics24/blob/master/worksheets/06_linearRegression/code/scripts/doLinearRegressionVisualCell.py}{doLinearRegressionVisualCell.py}
implements all the functionality required for this worksheet, but it has a few
lines that you will need to complete to make it work. In this exercise you will
perform all the mathematical operations required to estimate the regression
coefficients (e.g., matrix inversions, tranposes, and multiplications).
However, it is better to use numerical packages (e.g.,
\href{https://scipy.org/}{scipy}) as they take care of important numerical
details.

The first section you will have to complete is on file
\href{https://github.com/joacorapela/neuroinformatics24/blob/master/worksheets/06_linearRegression/code/scripts/https://github.com/joacorapela/neuroinformatics24/blob/master/worksheets/06_linearRegression/code/scripts/doLinearRegressionVisualCell.py}{doLinearRegressionVisualCell.py}:

\begin{lstlisting}[language=python]
    # calculate regression coefficients with train data using L2 regularisation
    X = utils.buildDataMatrix(px=px, order=order, nRDs=n_RDs)
    X_train, X_test, Y_train, Y_test = \
        sklearn.model_selection.train_test_split(X, Y,
                                                 test_size=test_percentage)
    I = np.eye(X.shape[1])
    coefs = ...
\end{lstlisting}

Lines 3-5 partition the data into train and test sets. We will use the train
set to estimate model parameters and the test set to asses the model predictive
power without overfitting.

You will need to complete line 7 to calculate the coefficients using the train
set and the formula given in class to compute regression coefficients using
L2 regularisation. You may want to use functions
\href{https://numpy.org/doc/stable/reference/generated/numpy.matmul.html}{np.matmul} for
matrix-matrix or matrix-vector multiplication,
\href{https://numpy.org/doc/stable/reference/generated/numpy.transpose.html}{np.transpose}
to transpose an array,
\href{https://numpy.org/doc/stable/reference/generated/numpy.linalg.inv.html}{np.linalg.inv}
for matrix inversion and/or
\href{https://numpy.org/doc/stable/reference/generated/numpy.linalg.solve.html}{np.linalg.solve}
to solve systems of equations.

The second section you will have to complete is on file
\href{https://github.com/joacorapela/neuroinformatics24/blob/master/worksheets/06_linearRegression/code/scripts/https://github.com/joacorapela/neuroinformatics24/blob/master/worksheets/06_linearRegression/code/scripts/doLinearRegressionVisualCell.py}{doLinearRegressionVisualCell.py}:

\begin{lstlisting}[language=python]
      # calculate residuals
      fitted_train = ...
      residuals_train = Y_train - fitted_train

      # compute correlation coefficient on test data
      fitted_test = ...
      rho_test = np.corrcoef(Y_test, fitted_test)[0, 1]
\end{lstlisting}

Given the coefficients estimated above, here you need to compute the fitted
response (or dependent variable) for the train and test segments. We use the
fitted train response to calculate the residuals for the train data. We use the
fitted test response to calculate their correlation coefficients with the
observed test dependent variable.

After you complete the missing parts of the code, you can run it with the
parameters set for the simulated complex cell specified in the shell script
\href{https://github.com/joacorapela/neuroinformatics24/blob/master/worksheets/06_linearRegression/code/scripts/doAnalyzeSimulatedComplexCell.csh}{doAnalyzeSimulatedComplexCell.csh}.
In this script you can specify the number of relevant dimensions (parameter
\texttt{--nRDs}) and the polynomial order (parameter \texttt{--order}) that you
want to use. Running this script with two relevant dimensions and with a
polynomial order 2 (i.e., parameters \texttt{--order=2 --nRDs=2}) I obtained the
estimated relevant dimensions in Figure~\ref{fig:analysis_simCC_rds}, the coefficients
in Figure~\ref{fig:analysis_simCC_coefs}, the histogram of residuals in
Figure~\ref{fig:analysis_simCC_histResiduals} and the fitted responses in
Figure~\ref{fig:analysis_simCC_fittedRes}.

\begin{figure}[H]
    \begin{center}
        \href{https://www.gatsby.ucl.ac.uk/~rapela/neuroinformatics/2023/ws6/figures/simCC_order4_nRDs2_regCoef0.0_RDs.html}{\includegraphics[width=5.5in]{figures/simCC_order4_nRDs2_regCoef0.0_RDs.png}}

        \caption{Relevant dimensions estimated for simulated data of a complex cell. Compare with Figure~\ref{fig:rdsSim}}.

        \label{fig:analysis_simCC_rds}
    \end{center}
\end{figure}

\begin{figure}[H]
    \begin{center}
        \href{https://www.gatsby.ucl.ac.uk/~rapela/neuroinformatics/2023/ws6/figures/simCC_order2_nRDs2_regCoef0.0_coefs.html}{\includegraphics[width=5.5in]{figures/simCC_order2_nRDs2_regCoef0.0_coefs.png}}

        \caption{Coefficients estimated for simulated data of a complex cell.
        Only the coefficients with indices 2 and 4 in the plot, corresponding to $\beta_{11}$ and
        $\beta_{22}$ in Eq.~\ref{eq:2RDsOrder3}, are significantly different from
        zero. This is the correct solution, as it can be seen from the model in
        Eq.~\ref{eq:energyModel}}

        \label{fig:analysis_simCC_coefs}
    \end{center}
\end{figure}

\begin{figure}[H]
    \begin{center}
        \href{https://www.gatsby.ucl.ac.uk/~rapela/neuroinformatics/2023/ws6/figures/simCC_order2_nRDs2_regCoef0.0_residuals.html}{\includegraphics[width=5.5in]{figures/simCC_order2_nRDs2_regCoef0.0_residuals.png}}

        \caption{Histogram of train residuals for the simulated complex cell (blue) and density of Normal distribution with the same mean and variance as that of the train residuals (red).}

        \label{fig:analysis_simCC_histResiduals}
    \end{center}
\end{figure}

\begin{figure}[H]
    \begin{center}
        \href{https://www.gatsby.ucl.ac.uk/~rapela/neuroinformatics/2023/ws6/figures/simCC_order2_nRDs2_regCoef0.0_predictions.html}{\includegraphics[width=5.5in]{figures/simCC_order2_nRDs2_regCoef0.0_predictions.png}}

        \caption{Cell responses versus predictions for the simulated complex cell. Their correlation coefficient appears in the title.}

        \label{fig:analysis_simCC_fittedRes}
    \end{center}
\end{figure}

Vary the number of relevant dimensions (parameter \texttt{--nRDs} in
\href{https://github.com/joacorapela/neuroinformatics24/blob/master/worksheets/06_linearRegression/code/scripts/doAnalyzeSimulatedComplexCell.csh}{doAnalyzeSimulatedComplexCell.csh})
and model order (parameter \texttt{--order} in
\href{https://github.com/joacorapela/neuroinformatics24/blob/master/worksheets/06_linearRegression/code/scripts/doAnalyzeSimulatedComplexCell.csh}{doAnalyzeSimulatedComplexCell.csh})
to select their optimal values. Plot
Figure~\ref{fig:analysis_simCC_rds}-\ref{fig:analysis_simCC_fittedRes} for the
best and a poor set of parameters.

\section{Real complex cell}

The shell script
\href{https://github.com/joacorapela/neuroinformatics24/blob/master/worksheets/06_linearRegression/code/scripts/doAnalyzeRealComplexCell.csh}{doAnalyzeRealComplexCell.csh}
contains the parameters required to characterise a complex cell in primary
visual cortex of an anaesthetised cat~\citep{felsenEtAl05}. Find the best set of
parameters to characterise this cell. Plot
Figure~\ref{fig:analysis_simCC_rds}-\ref{fig:analysis_simCC_fittedRes} for the
best and a poor set of parameters.


\bibliographystyle{plainnat}
\bibliography{receptiveFields,stats}

\end{document}
